\section{Thực nghiệm với Dữ liệu Phân loại (Categorical Attributes)}

\subsection{Mục tiêu của Thực nghiệm}

Hầu hết các thuật toán phân cụm (như K-means hay DBSCAN gốc) hoạt động dựa trên khoảng cách hình học (Euclidean distance), do đó chúng gặp khó khăn khi xử lý dữ liệu dạng phân loại (ví dụ: Nghề nghiệp, Tình trạng hôn nhân).

\textbf{Mục tiêu:} Tác giả muốn chứng minh rằng:
\begin{enumerate}
    \item FairDen có thể xử lý trực tiếp dữ liệu phân loại nhờ công thức khoảng cách hỗn hợp.
    \item Việc thêm dữ liệu phân loại vào không làm giảm chất lượng clustering, mà ngược lại còn giúp tăng độ công bằng.
\end{enumerate}

\subsection{Thiết lập Thực nghiệm}

Tác giả so sánh hai phiên bản của thuật toán FairDen trên cùng bộ dữ liệu:

\begin{itemize}
    \item \textbf{FairDen- (FairDen Minus):} Chỉ sử dụng các cột số (Numerical attributes). Loại bỏ hoàn toàn các cột phân loại.
    \item \textbf{FairDen (Full):} Sử dụng tất cả các cột (cả số và phân loại). Sử dụng công thức khoảng cách hỗn hợp kết hợp Euclidean (cho số) và độ đo Goodall (cho phân loại).
\end{itemize}

\textbf{Lưu ý về thước đo:} DCSI không được định nghĩa cho dữ liệu có thuộc tính phân loại vì nó dựa trên khoảng cách hình học thuần túy. Do đó, tác giả đánh giá bằng $\text{ARI}_{DB}$ và $\text{NMI}_{DB}$ --- đo mức độ tương đồng giữa kết quả của FairDen và DBSCAN.

\subsection{Kết quả Thực nghiệm}

Bảng \ref{tab:categorical_results} so sánh kết quả khi loại trừ/bao gồm (FairDen-/FairDen) thuộc tính phân loại cho các tập dữ liệu Adult (sensitive: gender/race), Bank (sensitive: marital) và Student (sensitive: address). Lưu ý rằng các đối thủ cạnh tranh không thể xử lý thuộc tính phân loại.

\begin{table}[H]
\centering
\caption{So sánh kết quả khi loại trừ/bao gồm (FairDen-/FairDen) thuộc tính phân loại. Lưu ý rằng các đối thủ fair clustering khác không thể xử lý thuộc tính phân loại.}
\label{tab:categorical_results}
\begin{tabular}{l|l|c|c|c|c}
\toprule
& \textbf{Algorithm} & \textbf{Balance} & $\mathbf{ARI_{DB}}$ & $\mathbf{NMI_{DB}}$ & \textbf{Noise} \\
\midrule
\multirow{4}{*}{\rotatebox{90}{Adult (g)}}
& DBSCAN & 0.71 & 1.00 & 1.00 & 0.99 \\
& FairDen & \textbf{0.96} & \textbf{0.00} & \textbf{0.00} & \textbf{0.00} \\
& FairDen- & \underline{0.86} & \underline{0.00} & \underline{0.00} & \underline{0.00} \\
& Ground Truth & 0.66 & -0.04 & 0.01 & -- \\
\midrule
\multirow{4}{*}{\rotatebox{90}{Adult (r)}}
& DBSCAN & 0.50 & 1.00 & 1.00 & 0.01 \\
& FairDen & \textbf{0.86} & \textbf{0.01} & \textbf{0.01} & \textbf{0.00} \\
& FairDen- & \underline{0.83} & \underline{0.01} & \underline{0.02} & \underline{0.00} \\
& Ground Truth & 0.52 & 0.01 & 0.01 & -- \\
\midrule
\multirow{4}{*}{\rotatebox{90}{Bank (m)}}
& DBSCAN & 0.79 & 1.00 & 1.00 & 0.00 \\
& FairDen & \textbf{0.99} & \textbf{0.01} & \textbf{0.01} & \textbf{0.00} \\
& FairDen- & \underline{0.98} & \underline{0.01} & \underline{0.01} & \underline{0.00} \\
& Ground Truth & 0.86 & 0.00 & 0.00 & -- \\
\midrule
\multirow{4}{*}{\rotatebox{90}{Student (a)}}
& DBSCAN & 0.57 & 1.00 & 1.00 & 0.16 \\
& FairDen & \textbf{0.95} & \textbf{0.45} & \textbf{0.27} & \textbf{0.00} \\
& FairDen- & \underline{0.93} & \underline{0.52} & \underline{0.31} & \underline{0.00} \\
& Ground Truth & 0.58 & 0.00 & 0.08 & -- \\
\bottomrule
\end{tabular}
\end{table}

\subsection{Phân tích Kết quả}
\begin{itemize}
    \item \textbf{Balance Tăng lên khi sử dụng Categorical:}
    Trên tất cả các tập dữ liệu, \textbf{FairDen} đạt Balance cao hơn \textbf{FairDen-}Dữ liệu phân loại chứa nhiều thông tin xã hội quan trọng (nghề nghiệp, trình độ học vấn). Khi thuật toán ``biết'' thêm các thông tin này, nó có thêm cơ sở để gom nhóm các đối tượng tương đồng, từ đó việc chia đều các nhóm nhạy cảm trở nên tự nhiên hơn.

    \item \textbf{Cấu trúc Mật độ được Bảo toàn:}
    Chỉ số $\text{ARI}_{DB}$ và $\text{NMI}_{DB}$ của FairDen và FairDen- là \textbf{xấp xỉ nhau}. Có nhiều lo ngại rằng thêm dữ liệu phân loại sẽ làm hỏng cấu trúc hình học. Tuy nhiên, kết quả này chứng minh rằng việc thêm categorical \textbf{không gây ảnh hưởng tiêu cực} đến tính liên thông mật độ.

    \item \textbf{Nhiễu Thấp:} Cả FairDen và FairDen- đều có Noise = 0.00 trên hầu hết các tập dữ liệu. Thuật toán không bị gây nhiễu bởi dữ liệu hỗn hợp và vẫn tự tin phân loại đa số các điểm vào các cụm chính thức.

    \item \textbf{So sánh với tác giả:} Kết quả thực nghiệm của nhóm tương đồng với bài báo gốc --- FairDen với categorical attributes đạt Balance cao hơn FairDen- trên tất cả các tập dữ liệu. Nhóm bổ sung thêm tập \textbf{Student Performance} với thuộc tính nhạy cảm \textit{address} (Urban/Rural), và xu hướng tương tự được tái hiện.
\end{itemize}
